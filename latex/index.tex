
\begin{DoxyItemize}
\item {\bfseries Autores\+:} Oliver Gibson (\href{mailto:gibson31@gmail.com}{\tt gibson31@gmail.\+com}), Catalina Juarros (\href{mailto:catalinajuarros@gmail.com}{\tt catalinajuarros@gmail.\+com}), Miguel Nehmad Alché (\href{mailto:mikealche@gmail.com}{\tt mikealche@gmail.\+com}), Jessica Singer (\href{mailto:singer.jeess@gmail.com}{\tt singer.\+jeess@gmail.\+com})
\item {\bfseries Materia\+:} Algoritmos y Estructuras de Datos II
\end{DoxyItemize}

El presente documento describe la implementación de un módulo diccionario dentro del marco de Algoritmos y Estructuras de Datos II. Este módulo corresponde al T\+P2 del primer cuatrimestre del 2017 y su objetivo es servir de referencia para los docentes y de base para el enunciado del TP.

El trabajo práctico consiste en la implementación de un diccionario sobre árboles red-\/black, cuya descripción se puede encontrar en \cite{CormenLeisersonRivestStein2009}. La interfaz solicitada corresponde (vagamente) a la provista por C++03, a fin de poder comparar fácilmente los resultados obtenidos con los esperados.

\begin{DoxyParagraph}{Índice}

\end{DoxyParagraph}

\begin{DoxyItemize}
\item \hyperlink{Enunciado}{Enunciado}
\item \hyperlink{Interfaz}{Descripción de la interfaz}
\item \hyperlink{Implementacion}{Implementación del árbol red-\/black}
\item \hyperlink{Aliasing}{Aspectos de aliasing y uso de punteros}
\item \hyperlink{Castellano}{Uso del lenguaje coloquial}
\item \hyperlink{classaed2_1_1map}{Documentación del diccionario }
\item \hyperlink{axiomas}{Axiomas y proposiciones auxiliares}
\item \hyperlink{citelist}{Referencias bibliográficas} 
\end{DoxyItemize}