En esta pagina, y por conveniencia, se listan todos los axiomas y proposiciones auxiliares requeridos para formalizar los invariantes de representación y las funciones de abstracción. Previamente se presentan los renombres de los tipos usados.

\begin{DoxyParagraph}{Renombres de tipos}

\end{DoxyParagraph}

\begin{DoxyItemize}
\item Node es tupla(child\+: arreglo\mbox{[}2\mbox{]} de puntero(\+Node), parent\+: puntero(\+Node), color\+: Color, value\+: Value)
\item Value es Maybe(value\+\_\+type)
\item value\+\_\+type es tupla(clave\+: Key, significado\+: Meaning)
\end{DoxyItemize}

El T\+AD Maybe( $\alpha$) representa un tipo $\alpha$ extendido con un valor nulo. En otras palabras, el T\+AD Maybe se puede usar para representar los valores de los nodos, donde el nodo cabecera no tiene valor y los nodos internos sí tienen valor. (Esto independientemente de si se implementa con herencia o con un puntero o de otra forma.) Tiene dos observadores\+:
\begin{DoxyItemize}
\item nothing?(x)\+: que indica si x tiene un valor nulo, y
\item data(x)\+: que devuelve el valor de x, suponiendo que no es inválido.
\end{DoxyItemize}

La especificación de este T\+AD queda como ejercicio (no obligatorio).\hypertarget{axiomas_sec-Axiomas}{}\subsection{Axiomas y proposiciones auxiliares}\label{axiomas_sec-Axiomas}
En esta sección se deben incluir todos los axiomas y proposiciones auxiliares que se usen para describir los invariantes de representación, las funciones de abstracción, las precondiciones y las postcondiciones.

\begin{DoxyRemark}{Comentarios}
Recordar incluir un alias en el archivo doxyfile a fin de poder referenciar automaticamente a cada axioma desde las otras páginas.
\end{DoxyRemark}
Se muestran algunos ejemplos a continuación.

\begin{DoxyParagraph}{es\+Diccionario?}
Retorna true si la secuencia representa un diccionario

es\+Diccionario?\+: secu(tupla( $\alpha$, $\beta$)) $\to$ bool~\newline
es\+Diccionario?(s) $\equiv$ sin\+Repetidos?(\href{axiomas.html#primeros}{\tt primeros}(s)) 
\end{DoxyParagraph}


\begin{DoxyParagraph}{primeros}
Proyecta las primeras componentes de una secuencia de pares

primeros\+: secu(tupla( $\alpha$, $\beta$)) $\to$ secu( $\alpha$)~\newline
primeros(s) $\equiv$ {\bfseries if} vacia?(s) {\bfseries then} $<$$>$ {\bfseries else} $\pi_1$(prim(s)) $\bullet$ \href{axiomas.html#primeros}{\tt primeros}(fin(s)) {\bfseries fi} 
\end{DoxyParagraph}


\begin{DoxyParagraph}{esÁrbol?}
Proposición que dice si una estructura representada con punteros a nodo se corresponde efectivamente con un árbol binario finito. La primera proposición asegura que el árbol termina, mientras que la segunda afirma que ningún nodo es hijo de dos nodos distintos (es decir, no hay ciclos).

esÁrbol?\+: puntero(nodo) $\to$ bool ~\newline
esÁrbol?(p) $\equiv$ true $\Leftrightarrow$ ( $\exists$ k\+:nat) \textbackslash{}árbol\+K(p,k) $=_{\rm obs}$ árbolK(p,k+1) $\land$ \href{axiomas.html#sinRepetidosAB}{\tt sin\+Repetidos\+AB}(\textbackslash{}árbol\+K(p,k))


\end{DoxyParagraph}
\begin{DoxyParagraph}{árbolK}
Devuelve los primeros k niveles del árbol binario de punteros cuya raíz es p.

árbolK\+: puntero(nodo) $\times$ nat $\to$ ab(puntero(nodo)) ~\newline
árbol\+K(p,k) $\equiv$ ~\newline
 {\bfseries if} p = nullptr {\bfseries then} ~\newline
 nil ~\newline
 {\bfseries else} ~\newline
{\bfseries if} k = 0 {\bfseries then} ab(nil,p,nil) {\bfseries else} ab(\textbackslash{}árbolK(p-\/$>$child\mbox{[}0\mbox{]},k-\/1),p,\textbackslash{}árbolK(p-\/$>$child\mbox{[}1\mbox{]},k-\/1)) {\bfseries fi} ~\newline
 {\bfseries fi} 


\end{DoxyParagraph}
\begin{DoxyParagraph}{sin\+Repetidos\+AB}
Dice si un árbol binario tiene o no elementos repetidos.

sin\+Repetidos\+AB\+: ab( $\alpha$) $\to$ bool ~\newline
sin\+Repetidos\+A\+B(a) $\equiv$ ~\newline
{\bfseries if} a = nil {\bfseries then} ~\newline
true ~\newline
{\bfseries else} ~\newline
 $\_linebr eg$ (á?(raíz(a),izq(a)) $\lor$ á?(raíz(a),der(a))) $\land$ \href{axiomas.html#sinRepetidosAB}{\tt sin\+Repetidos\+AB}(izq(a)) $\land$ \href{axiomas.html#sinRepetidosAB}{\tt sin\+Repetidos\+AB}(der(a)) ~\newline
 {\bfseries fi} 


\end{DoxyParagraph}
\begin{DoxyParagraph}{está?}
Dice si un elemento está o no en un árbol binario.

está?\+:  $\alpha$ $\times$ ab( $\alpha$) $\to$ bool está?(e,a) $\equiv$ ~\newline
{\bfseries if} a = nil {\bfseries then} ~\newline
false ~\newline
{\bfseries else} ~\newline
raíz(a) = e $\lor$ á?(e,izq(a)) $\lor$ á?(e,der(a)) ~\newline
 {\bfseries fi} 


\end{DoxyParagraph}
\begin{DoxyParagraph}{es\+A\+B\+B\+Dicc}
Dice si el árbol binario que tiene a p como raíz es un A\+BB sin claves repetidas. Para asegurar esto último, las funciones auxiliares \href{axiomas.html#todosMenores}{\tt todos\+Menores} y \href{axiomas.html#todosMayores}{\tt todos\+Mayores} hacen comparaciones {\itshape estrictas}.

es\+A\+B\+B\+Dicc\+: puntero(nodo) $\to$ bool ~\newline
es\+A\+B\+B\+Dicc(p) $\equiv$ ~\newline
{\bfseries if} p = nullptr {\bfseries then} ~\newline
true ~\newline
{\bfseries else} ~\newline
\href{axiomas.html#todosMenores}{\tt todos\+Menores}(p-\/$>$child\mbox{[}0\mbox{]},p-\/$>$value.\+first) $\land$ \href{axiomas.html#todosMayores}{\tt todos\+Mayores}(p-\/$>$child\mbox{[}1\mbox{]},p-\/$>$value.\+first) $\land$ \href{axiomas.html#esABBDicc}{\tt es\+A\+B\+B\+Dicc}(p-\/$>$child\mbox{[}0\mbox{]}) $\land$ \href{axiomas.html#esABBDicc}{\tt es\+A\+B\+B\+Dicc}(p-\/$>$child\mbox{[}1\mbox{]}) ~\newline
 {\bfseries fi} 


\end{DoxyParagraph}
\begin{DoxyParagraph}{todos\+Menores}
Dice si todas las claves del árbol binario que tiene como raíz a p son estrictamentente menores a e.

todos\+Menores\+: puntero(nodo) $\times$ Key $\to$ bool ~\newline
todos\+Menores(p,e) $\equiv$ ~\newline
{\bfseries if} p = nullptr {\bfseries then} ~\newline
true ~\newline
{\bfseries else} ~\newline
p-\/$>$value.\+first $<$ e $\land$ \href{axiomas.html#todosMenores}{\tt todos\+Menores}(p-\/$>$child\mbox{[}0\mbox{]},e) $\land$ \href{axiomas.html#todosMenores}{\tt todos\+Menores}(p-\/$>$child\mbox{[}1\mbox{]},e) ~\newline
 {\bfseries fi} 


\end{DoxyParagraph}
\begin{DoxyParagraph}{todos\+Mayores}
Dice si todas las claves del árbol binario que tiene como raíz a p son estrictamentente mayores a e.

todos\+Mayores\+: puntero(nodo) $\times$ Key $\to$ bool ~\newline
todos\+Mayores(p,e) $\equiv$ ~\newline
{\bfseries if} p = nullptr {\bfseries then} ~\newline
true ~\newline
{\bfseries else} ~\newline
p-\/$>$value.\+first $>$ e $\land$ \href{axiomas.html#todosMayores}{\tt todos\+Mayores}(p-\/$>$child\mbox{[}0\mbox{]},e) $\land$ \href{axiomas.html#todosMayores}{\tt todos\+Mayores}(p-\/$>$child\mbox{[}1\mbox{]},e) ~\newline
 {\bfseries fi} 
\end{DoxyParagraph}
