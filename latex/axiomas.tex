En esta pagina, y por conveniencia, se listan todos los axiomas y proposiciones auxiliares requeridos para formalizar los invariantes de representación y las funciones de abstracción. Previamente se presentan los renombres de los tipos usados.

\begin{DoxyParagraph}{Renombres de tipos}

\end{DoxyParagraph}

\begin{DoxyItemize}
\item Node es tupla(child\+: arreglo\mbox{[}2\mbox{]} de puntero(\+Node), parent\+: puntero(\+Node), color\+: Color, value\+: Value)
\item Value es Maybe(value\+\_\+type)
\item value\+\_\+type es tupla(clave\+: Key, significado\+: Meaning)
\end{DoxyItemize}

El T\+AD Maybe( $\alpha$) representa un tipo $\alpha$ extendido con un valor nulo. En otras palabras, el T\+AD Maybe se puede usar para representar los valores de los nodos, donde el nodo cabecera no tiene valor y los nodos internos sí tienen valor. (Esto independientemente de si se implementa con herencia o con un puntero o de otra forma.) Tiene dos observadores\+:
\begin{DoxyItemize}
\item nothing?(x)\+: que indica si x tiene un valor nulo, y
\item data(x)\+: que devuelve el valor de x, suponiendo que no es inválido.
\end{DoxyItemize}

La especificación de este T\+AD queda como ejercicio (no obligatorio).\hypertarget{axiomas_sec-Axiomas}{}\subsection{Axiomas y proposiciones auxiliares}\label{axiomas_sec-Axiomas}
En esta sección se deben incluir todos los axiomas y proposiciones auxiliares que se usen para describir los invariantes de representación, las funciones de abstracción, las precondiciones y las postcondiciones.

\begin{DoxyRemark}{Comentarios}
Recordar incluir un alias en el archivo doxyfile a fin de poder referenciar automaticamente a cada axioma desde las otras páginas.
\end{DoxyRemark}
Se muestran algunos ejemplos a continuación.

\begin{DoxyParagraph}{es\+Diccionario?}
Retorna true si la secuencia representa un diccionario

es\+Diccionario?\+: secu(tupla( $\alpha$, $\beta$)) $\to$ bool~\newline
es\+Diccionario?(s) $\equiv$ sin\+Repetidos?(\href{axiomas.html#primeros}{\tt primeros}(s)) 
\end{DoxyParagraph}


\begin{DoxyParagraph}{primeros}
Proyecta las primeras componentes de una secuencia de pares

primeros\+: secu(tupla( $\alpha$, $\beta$)) $\to$ secu( $\alpha$)~\newline
primeros(s) $\equiv$ {\bfseries if} vacia?(s) {\bfseries then} $<$$>$ {\bfseries else} $\pi_1$(prim(s)) $\bullet$ \href{axiomas.html#primeros}{\tt primeros}(fin(s)) {\bfseries fi} 
\end{DoxyParagraph}
